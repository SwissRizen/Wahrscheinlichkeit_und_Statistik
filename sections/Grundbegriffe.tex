\section{Grundbegriffe}
\subsection{Probability Space}
\begin{mainbox}{Axiome von Kolmogorov}
    Das Tuple $(\Omega, \A, \P)$ ist ein \textbf{Probability Space} mit 
    \begin{enumerate}[label=\Roman*.]
        \item \textbf{Grundraum} $\Omega$ mit $\Omega \neq \emptyset$, wobei $\omega \in \Omega$ ein Elementarereignis ist.
        \item \textbf{$\sigma$-Algebra} $\A \subseteq \Pset(\Omega)$ wobei gilt:
        \begin{enumerate}[label=\arabic*.]
            \item $\Omega \in \A$
            \item $A \in \A \implies A^\complement \in \A$
            \item $A_1,A_2,\dots \in \A \implies \bigcup_i A_i \in \A$
        \end{enumerate}
        \item \textbf{Wahrscheinlichkeitsmass} $\P$ auf $(\Omega, \A)$ ist eine Abbildung $\P: \A \mapsto [0,1]$, wobei gilt:
        \begin{enumerate}[label=\arabic*.]
            \item $\P(\Omega) = 1$
            \item $A_1, A_2, \dots \in \A, \forall i \neq j: A_i \cap A_j = \emptyset \\ \implies \P(\bigcup_i A_i) = \sum_{i= 1}^\infty \P(A_i)$
        \end{enumerate}
    \end{enumerate}
\end{mainbox}

\textbf{De-Morgan}
    
Sei $(A_i)_{i \geq 1}$ eine Folge von beliebigen Mengen. Dann gilt
    $$\left(\bigcup_{i = 1}^{\infty}A_i\right)^\complement = \bigcap_{i = 1}^{\infty} (A_i)^\complement$$


Daraus folgt
\begin{enumerate}[label=\arabic*.]
    \item $A_1,A_2, \dots \in \A \implies \bigcap_{i=1}^{\infty}A_i \in \A$
    \item $A,B \in \A \implies (A \cup B), (A \cap B) \in \A$
\end{enumerate}
und für $A, B \in \A$
\begin{enumerate}[label=\arabic*.]
    \item $\P(A^\complement) = 1 - \P(A)$
    \item $A \subseteq B \implies \P(A) \leq \P(B)$
    \item $\P(A \cup B) = \P(A)+\P(B)-\P(A \cap B)$
\end{enumerate}
Sei $A_1,A_2, \dots \in \A$, dann gilt:\\
\textbf{Union Bound}
    $$\P\left(\bigcup_{i=1}^{\infty} A_i \right) \leq \sum_{i=1}^{\infty}\P(A_i)$$
\textbf{Siebformel}
    $$\P\left(\bigcup_{i=1}^{n}A_i\right) = \sum_{k = 1}^{n}(-1)^{k+1}\sum_{1\leq i_1 < \dots < i_k \leq n} \P(A_{i_1} \cap \dots \cap A_{i_k})$$

    \textbf{Atome}

        Sei $\Omega$ nicht leer und discrete. Sei $\F$ eine beliebige $\sigma$-Algebra über $\Omega$. 
    
        Eine nichtleere Menge $A \in \F$ heisst \textbf{atomare} Menge von $\F$ falls für alle $B \in \F$ gilt:
        $$B \subseteq A \implies B = \emptyset \lor B = A$$
        (Intuitiv: $A$ ist die kleinste nichtleere Menge bezüglich der Inklusion in $\F$)
    
        Die Menge der atomaren Mengen von $\F$ bezeichnen wir mit Atom$(\F)$.
    
        Jedes Element von $\F$ lässt sich als countablee Vereinigung von Elementen aus Atom$(\F)$ schreiben.
    
\subsection{Conditional Probability}
 Sei $(\Omega, \A, \P)$ ein Probability Space.
 \begin{mainbox}{Conditional Probability}
    Sei $A, B \in \A$ und $\P(B) > 0$, dann ist die \textbf{bedingte Wahrscheinlichkeit von $A$ gegeben $B$}
    $$\P(A|B) = \frac{\P(A \cap B)}{\P(B)}$$
 \end{mainbox}
\textbf{Law of total Probability}

Sei $(B_i)_{i\in I}$ eine Partition von $\Omega$. Dann gilt für jedes beliebige $A \in \A$
$$\P(A) = \sum_{i:\ \P(B_i)>0} \P(A|B_i)\P(B_i)$$
\textbf{Bayes theorem}

Aus der Definition der bedingten Prob. folgt sofort die Bayessche Formel, welche den Zusammenhang zwischen $\P(A|B)$ und $\P(B|A)$ beschreibt:
$$\P(B|A) = \frac{\P(A|B)\P(B)}{\P(A)}$$
Mit dem \textbf{Law of total Prob.} können wir $\P(A)$ umschreiben und kommen auf folgende Form:

Sei $(B_i)_{i\in I}$ eine \textbf{Partition} von $\Omega$. Dann gilt für jedes beliebige $A \in \A, \P(A) > 0$
$$\P(B_i|A) = \frac{\P(A|B_i)\cdot\P(B_i)}{\sum_{j:\ \P(B_j)>0}\P(A|B_j)\cdot \P(B_j)}$$

\textbf{Intuition Bayessche Statistik}

In dieser Form würde man $A$ als das \textbf{eingetretene Event} und die $B_i$ als die verschiedene \textbf{Hypothesen} verstehen. 

In der Bayesschen Statistik versucht man die Hypothese zu finden, so dass $\P(B_i|A)$ \textbf{maximiert} wird.

(Wurde in der Vorlesung nicht weiter behandelt)
\subsection{Independence} 
\begin{mainbox}{}
    Zwei Eventse $A, B \in \A$ heissen \textbf{independent}, wenn 
    $$\P(A\cap B) = \P(A) \cdot \P(B)$$
\end{mainbox}
\begin{itemize}
    \item $\P(A) \in \{0, 1\} \implies A \text{ zu jedem Event independent}$
    \item $A$ zu sich selbst independent $\implies \P(A) \in \{0,1\}$
    \item $A, B$ independent $\implies$ $A, B^\complement$ independent
\end{itemize}
Wenn $\P(A) > 0, \P(B) > 0$ gilt:\\
 $A,B$ independent $\iff$ $\P(A|B)=\P(A)$ $\iff$ $\P(B|A)=\P(B)$
\\
\\
Wir können die Definition der Independence auf beliebige Mengen von Eventsen erweitern.
\begin{mainbox}{}
    Eine Kollektion von Eventsen $(A_i; i \in I)$ heisst \textbf{(stochastisch) independent}, wenn 
    $$J \subseteq I \text{ finite} \implies \P\left(\bigcap_{i\in J} A_i\right) = \prod_{i \in J} \P(A_i)$$
\end{mainbox}

