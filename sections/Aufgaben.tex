\section{Aufgaben}
\subsection{Multiple Choice}

Seien \(X,Y\) zwei RV mit Jointr Density \(f_{X,Y}\). Welche Aussage ist korrekt?
\begin{itemize}
	\item[\checkmark] \(X,Y\) sind immer continuous
	\item[\(\square\)] Die RV sind nicht notwendigerweise continuous.
\end{itemize}

Sei $Y$ eine continuouse Random Variable. Für alle $s, t \in \R^+$:

$\exists \lambda > 0. \ Y \sim Exp(\lambda) \iff \P(Y > s) = \P(Y > s + t \mid Y > t)$ 
\begin{itemize}
	\item[\checkmark] wahr.
	\item[\(\square\)] falsch.
\end{itemize}


% \noindent
% Seien \((X_i)_{i = 1}^n\) uiv. mit Cum. Distribution Function \(F_{X_i} = F\). Was ist die Cum. Distribution Function von \(M = \text{max}(X_1,...,X_n)\)?
% \begin{itemize}
% 	\item[\checkmark] \(F_M(a) = F(a)^n\)
% 	\item[\(\square\)] \(F_M(a) = 1 - F(a)^n\)
% 	\item[\(\square\)] \(F_M(a) = (1 - F(a))^n\)
% \end{itemize}

% \noindent
% Seien \(X, Y\) independent und lognormalverteilt (\(\ln X, \ln Y\) sind normalverteilt). Welche Aussage ist korrekt?
% \begin{itemize}
% 	\item[\checkmark] \(XY\) ist lognormalverteilt
% 	\item[\(\square\)] \(XY\) ist normalverteilt
% 	\item[\(\square\)] \(e^{X + Y}\) ist normalverteilt
% \end{itemize}

\subsection{Aufgaben Wahrscheinlichkeit}
\subsubsection*{\texorpdfstring{Density von \(\max(X_1,X_2)\)}{Density von max()}}

Seien \(X_1, X_2 \sim \mathcal{U}[0,1]\) independente RV und sei \(X = \max (X_1, X_2)\). Berechne die Densityfunktion von \(X\) und \(\P[X_1 \leq x \mid X \geq y]\).
\begin{align*}
	F_X(t) & = \P[\max(X_1, X_2) \leq t]                                                                                                              \\ &= \P[X_1 \leq t] \cdot \P[X_2 \leq t] = F_{X_1}(t) \cdot F_{X_2}(t) \\
	f_X(t) & = \frac{d}{dt} F_{X_1}(t) \cdot F_{X_2}(t) = \frac{d}{dt} t^2 \cdot \mathbb{I}_{0 \leq t \leq 1} = 2t \cdot \mathbb{I}_{0 \leq t \leq 1}
\end{align*}

Für die Wahrscheinlichkeit brauchen wir eine Fallunterscheidung: \smallskip

\begin{enumerate}
	\item \(x < 0\) oder \(1 < x\):
	      \[\mathbb{P}(X_1 \leq x \mid X \geq y) = 0 \text{ bzw. } 1\]
	\item \(0 \leq x \leq y \leq 1\):
	      \[\frac{\mathbb{P}(X_1 \leq x, X_2 \geq y)}{\mathbb{P}(X \geq y)} = \frac{x(1-y)}{1 - y^2}\]
	\item \(0 \leq y \leq x \leq 1\):
	      \[\frac{\mathbb{P}(X_1 < y, X_2 \geq y) + \P(y \leq X_1 \leq x)}{\mathbb{P}(X \geq y)} = \frac{x - y^2}{1 - y^2}\]
\end{enumerate}

\subsubsection*{Gemeinsame Density}

Bestimme die Joint Density von \(P \sim \mathcal{U}[0,1]\) und \(H \sim \mathcal{U}[0,P]\). Wir wissen:
\[f_P(p) = \mathbb I_{p \in [0,1]} \quad f_{H | P}(h \mid p) = \frac{1}{p} \cdot \mathbb{I}_{h \in [0,p]}\]

\noindent
Somit ist:
\[f_{P, H} (p, h) = f_P(p) \cdot f_{H | P}(h \mid p) = \frac{1}{p} \cdot \mathbb I_{0 \leq h \leq p \leq 1}\]

\subsubsection*{Maximum und Minimum gleichverteilter RVen}
Seien $U_1, U_2, U_3$ independente, $\mathcal{U}([0,1])$-verteilte Random Variables. 

Wir betrachten die continuousen RV $L := \min(U_1, U_2, U_3)$ und $M:=\max(U_1, U_2, U_3)$.

Zeige für beliebige $\phi, \psi: \R \to \R$ stückweise continuous und bounded, dass 
\[\E\left(\phi(M)\cdot\psi(L)\right) = \int\limits_{-\infty}\limits^\infty \int\limits_{-\infty}\limits^\infty \phi(m) \cdot \psi(l) \cdot6(m-l)\mathds{1}_{0\leq l\leq m \leq 1}\mathop{dl}\mathop{dm}\]
Wegen Independence ist die Joint Density durch $f(u_1, u_2, u_3) = \mathds{1}_{u_1 \in [0,1]}\mathds{1}_{u_2 \in [0,1]}\mathds{1}_{u_3 \in [0,1]}$ bestimmt.

\begin{align*}
	&\E(\phi(M)\psi(L)) \\
	&= \int_{\R^3} \phi(\max(u_1, u_2, u_3))\psi(\min(u_1, u_2, u_3)) f(u_1, u_2, u_3) \mathop{du_1}\mathop{du_2}\mathop{du_3}\\
	&= \int\limits_0\limits^1 \int\limits_0\limits^1 \int\limits_0\limits^1 \phi(\max(u_1, u_2, u_3))\psi(\min(u_1, u_2, u_3)) \mathop{du_1}\mathop{du_2}\mathop{du_3}
\end{align*}
Wir zerteilen jetzt dieses Integral in 6 Cases mit Indikatorfunktionen (die einzelnen Integrale summiert ergeben das gesuchte Integral). 
Beispielrechnung mit $\mathds{1}_{u_1 \leq u_2 \leq u_3}$ (andere Fälle analog).
\begin{align*}
	&= \int\limits_0\limits^1 \int\limits_0\limits^1 \int\limits_0\limits^1 \phi(\max(u_1, u_2, u_3))\psi(\min(u_1, u_2, u_3)) \mathds{1}_{u_1 \leq u_2 \leq u_3} \mathop{du_1}\mathop{du_2}\mathop{du_3}\\
	&= \int\limits_0\limits^1 \int\limits_0\limits^1 \int\limits_0\limits^1 \phi(u_3)\psi(u_1) \mathds{1}_{u_1 \leq u_2 \leq u_3} \mathop{du_1}\mathop{du_2}\mathop{du_3}\\
	&= \int_0^1 \phi(u_3) \left(\int_0^{u_3}\psi(u_1)\left(\int_{u_1}^{u_3}\mathop{du_2}\right)\mathop{du_1}\right)\mathop{du_3}\\
	&= \int_0^1\int_0^1 \phi(u_3)\psi(u_1)(u_3-u_1)\mathds{1}_{u_1\leq u_3}\mathop{du_1}\mathop{du_3}
\end{align*}
Die anderen 5 Fälle sind analog und deshalb folgt
\[\E\left(\phi(M)\cdot\psi(L)\right) = \int\limits_{-\infty}\limits^\infty \int\limits_{-\infty}\limits^\infty \phi(m) \cdot \psi(l) \cdot6(m-l)\mathds{1}_{0\leq l\leq m \leq 1}\mathop{dl}\mathop{dm}\]